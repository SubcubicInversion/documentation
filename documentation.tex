\documentclass[12pt, letterpaper]{article}

\usepackage[utf8]{inputenc}
\usepackage{array}
\usepackage{amsmath, amssymb, amsthm}
\usepackage{enumerate}
\usepackage{newunicodechar}
\usepackage{mathtools}
\usepackage{bm}

\DeclareUnicodeCharacter{2212}{-}

\theoremstyle{remark}
\newtheorem{problem}{Problem}

\theoremstyle{remark}
\newtheorem*{solution}{\textbf{Solution}}

\newenvironment{amatrix}[1]{%
	\left[\begin{array}{@{}*{#1}{c}|c@{}}
			}{                 %
		\end{array}\right]
}

\title{Subcubic Parallelized Matrix Inversion \\
	\large Through Strassen's Algorithm}
\author{Andres Valdes, Alejandro Perez, Dewey Morris}
\date{April, 2018}

\begin{document}
    \begin{titlepage}
        \maketitle
        \thispagestyle{empty}

    \end{titlepage}

    \begin{abstract}
        
        Matrix inversion is one of the most computationally complex 
        matrix operations. This is unfortunate, as matrix inversion 
        has many practical applications, including encryption, where the 
        contents of a matrix are obfuscated by multiplying by a square,
        invertible cypher matrix and can only be retrieved my multiplying
        it by the inverse of that cypher.

    \end{abstract}

    \pagebreak

    \section{Introduction}

    \noindent 
    A matrix is a rectangular array of numbers or expressions, for example: 
    \[\begin{bmatrix*}
        a & b & c & d \\
        e & f & g & h
    \end{bmatrix*}\]

    This is a what is referred to as a \(2\times{4}\) matrix, as it 
    has two rows and four columns. In general, we speak about matrices as 
    \(m\times{n}\), where \(m\) is the number of rows and \(n\) is the number 
    of columns. The types of matrices we will deal with are called "square" 
    matrices, this is the case that we have an \(m\times{n}\) matrix such that 
    \(m = n\), we refer to these also as \(n\times{n}\) matrices. Only square, 
    non-singular\footnote{A matrix whose determinant is nonzero.} 
    matrices are invertible.

    The inverse of a matrix \(A\), is a matrix \(A^{-1}\) that when multiplied by \(A\), 
    yields the identity matrix, \(I\). \(I_n\) is the square, \(n\times{n}\) 
    matrix with the top-leftmost to the bottom-rightmost diagonal filled in 
    with ones, and where every other element is a zero.
    \[I_n = \begin{bmatrix*}
        1 & 0 & 0 & \dots & 0 \\
        0 & 1 & 0 & \dots & 0 \\
        0 & 0 & 1 & \dots & 0 \\
        \vdots & \vdots & \vdots & \ddots & 0 \\
        0 & 0 & 0 & \dots & 1
    \end{bmatrix*}_{n\times n}\]

    So \(AA^{-1} = I\), this is the linear algebra equivalent of \(a\frac{1}{a} = 1\).

    \bigskip

    Traditionally, the inversion of a matrix involves the use of the formula
    \[A^{-1} = \frac{1}{det(A)}adj(A),\ adj(A) = (C_{A})^T\]

    Where \(C_A\) is the coefficient matrix of \(A\), that is, a matrix of 
    the dimension of \(A\) with determinants filled in for each of the elements.

    This is a particularly heavy computation, as the common algorithm for determinant
    computation is of the computation complexity \(O(n!)\), with the best algorithm 
    being \(O(n^3)\), forcing us to do \(O(n^5)\) computations for the coefficient 
    matrix alone, as there are \(n^2\) elements in \(A\).

    \bigskip

    Thankfully, we do have a way of reducing the problem for the calculation of 
    a matrix inverse via an analytic inversion formula known as blockwise inversion. 
    In blockwise inversion, we split our matrix \(M\) into matrix blocks, such that
    \[M^{-1} = \begin{bmatrix*}
        A & B \\
        C & D
    \end{bmatrix*}^{-1} = 
    \begin{bmatrix*}
        A^{-1} + A^{-1}B(D - CA^{-1}B)^{-1}CA^{-1} & -A^{-1}B(D - CA^{-1}B)^{-1} \\
        -(D - CA^{-1}B)^{-1}CA^{-1} & (D - CA^{-1}B)^{-1}
    \end{bmatrix*}\]
    
    The advantage to this divide-and-conquer algorithm is that it can be shown 
    that the computational complexity of the entire inversion is the same as the
    computational complexity of the multiplication algorithm used. This is where 
    Strassen's algorithm, with a complexity of \(O(n^{log_27})\), comes into play.

    \bigskip

    Strassen describes an algorithm to multiply two matrices by splitting
    each into four matrix blocks of equal dimensions and performing seven multiplications
    between those blocks as opposed to eight.
    \[A = \begin{bmatrix*}
        A_{11} & A_{12} \\
        A_{21} & A_{22}
    \end{bmatrix*},\ B =
    \begin{bmatrix*}
        B_{11} & B_{12} \\
        B_{21} & B_{22}
    \end{bmatrix*},\ AB = C =
    \begin{bmatrix*}
        C_{11} & C_{12} \\
        C_{21} & C_{22}
    \end{bmatrix*}\]

    We define seven new matrices
    \begin{align*}
        M_1 &\coloneqq (A_{11} + A_{22})(B_{11} + B_{22}) \\
        M_2 &\coloneqq (A_{21} + A_{22})B_{11} \\
        M_3 &\coloneqq A_{11}(B_{12} - B_{22}) \\
        M_4 &\coloneqq A_{22}(B_{21} - B_{11}) \\
        M_5 &\coloneqq (A_{11} + A_{12})B_{22} \\
        M_6 &\coloneqq (A_{21} - A_{11})(B_{11} + B_{12}) \\
        M_7 &\coloneqq (A_{12} - A_{22})(B_{21} + B_{22})
    \end{align*}

    Note that for each matrix, we multiply only once. We can 
    now compose these matrices into \(C\), or \(AB\).
    \begin{align*}
        C_{11} &= M_1 + M_4 - M_5 + M_7 \\
        C_{12} &= M_3 + M_5 \\
        C_{21} &= M_2 + M_4 \\
        C_{22} &= M_1 - M_2 + M_3 + M_6
    \end{align*}
    And thus, we've achieved matrix multiplication in subcubic time,
    and by extension, we can achieve matrix inversion in subcubic time.

\end{document}
